\documentclass[a4paper, 12pt]{article}
\usepackage[UTF8]{ctex}

\begin{document}
	\title{基础逻辑}
	\author{撰写人:朱善哲\\小组成员:马楷恒、王晟宇、李畅锦、朱博文}
	\date{2024年9月}
	\maketitle
	\section{敌我双方基础属性}
		\subsection{敌方}
			\begin{itemize}
				\item 生命值
				\item 防御力
				\item 法抗
				\item 移动速度
			\end{itemize}
		\subsection{我方}
			\begin{itemize}
				\item 攻击力
				\item 伤害类型(物理/法术/真实)
				\item 攻击间隔(攻击速度)
				\item 部署/升级费用
				\item 再部署时间
			\end{itemize}
	\section{战斗逻辑}
		\subsection{伤害计算原理}
			\subsubsection{物理伤害端}
				直接减法计算,由攻击者的攻击力减去受击者的防御力,差值为被攻击者受到的伤害;若攻击者攻击力小于受击者防御力,则受击者受到$5\%$攻击力的保底伤害
			\subsubsection{法术伤害端}
				作乘法运算,由攻击者的攻击力乘以$(100-$受击者法抗$)\%$
			\subsubsection{真实伤害端}
				不参与防御力$/$法抗计算
			\subsubsection{伤害结算顺序}
				$+/-$攻击力固定数值$>+/-$攻击力百分比$>*$攻击力百分比$>$防御力$/$法抗计算\\
				\indent 前两者结算结果会显示在“查看”面板上
		\subsection{敌方单位运动逻辑}
			以敌方单位中心点坐标作为判定标准,沿初设路线运动
		\subsection{攻击锁定、判定原理}
			\subsubsection{索敌机制}
				我方单位优先攻击攻击范围内优先级最高单位\\
				\indent 若有多个敌方单位优先级相同,则随机攻击\\
			\subsubsection{攻击优先级}
				我方单位:具有嘲讽敌方单位$>$剩余路径最短\\
			\subsubsection{判定机制}
				在攻击范围有可攻击目标后释放攻击物,当攻击物到达目标中心后消失,再进行伤害结算、后续反馈结算,若在攻击物飞行过程中目标消失则攻击物直接消失,不进行后续反馈结算\\
				\indent 若攻击为范围伤害,则范围内敌方单位同步进行结算\\
				\indent 地面单位与高台单位以单位中心点为判定标准,飞行单位以单位中心点在地面投影为判定点,格子边缘隶属于共享格子边缘的两格,但对相同伤害不进行两次判定结算
		\subsection{攻击速度、攻击间隔计算}
			敌我单位具有基础攻击间隔,基础攻击速度(默认100),实战中攻击间隔$=$基础攻击间隔$/$[(100$+$攻击速度加成)$/100$],转化为帧数四舍五入\\
			\indent 如:我方单位基础攻击间隔$=1.5s$,若受到buff攻速提高50,则实际攻击间隔为$1s$;若受到debuff攻速降低50,则实际攻击间隔为$3s$\\
			\indent 两次攻击之间间隔按上一次攻击结束时状态决定,即不受间隔期间攻速buff与debuff影响
		\subsection{部署、撤退与费用回复}
			部署我方单位时会显示可部署地块\\
			\indent 部署我放单位需要花费一定费用,费用一般会自然回复(30帧一点),若环境有降低费用回复效果则再进行乘算,如:若有$-75\%$费用回复效果,则费用自然回复为120帧一点\\
			\indent 可对已部署的我放单位进行撤退,若撤退则返还部署时所花费用的一半
		\subsection{敌方单位入点}
			敌方单位中心判定点进入防守点格子内(即到达格子边缘且未被阻挡或者控制在原地)即为入点
		\subsection{时间流逝}
			正常$1s$流失30帧,场内费用回复、技能技力回复、攻击间隔、移动速度等都换算为帧数,若切换为逐帧模式,则$1s$流失1帧

\end{document}