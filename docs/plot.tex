\documentclass[a4paper]{article}
\usepackage[UTF8]{ctex}

\begin{document}
	\title{剧情}
	\author{撰写人:朱善哲\\小组成员:马楷恒、王晟宇、李畅锦、朱博文}
	\date{2024年10月}
	\maketitle
	\section{}
		“呼……呼……”他踏上山崖,揉搓着双手试图挤出一丝温暖。他已经可以望见,此行的目的地。再往前走,就是久无人踏入之境————北部冰原。在此前,不知多少科考队和探险家试图揭开这片神秘领域的面纱,但是鲜有人能够活着回来。即便是回来的人,一个个都变得疯疯癫癫,没有人能推测出他们想表达什么,而且他们无一例外的没过两三天就死于病榻之上。医者术士推测冰原的经历使他们染上某种疾病,但是为何患上,如何处理,大家都不能知晓。\\
		\indent 若是这片冰原一如往日倒也在大家的接受范围之内,但是在近几个月,北部冰原的边境似乎一直在南扩,生活在边境地区的人们越来越多的患上这种病症。搞清楚这片冰原的秘密似乎已经变得刻不容缓,但是大家也怕因此丧命。\\
		\indent “所以……他们派你来了……”边境部落的长老缓缓开口,银白的胡须在寒风中颤巍。他无声的点了点头。此行无异于送死,两人都心知肚明,但是临行前言“死”肯定不是一个好兆头,因此两人都未多说。长老相信,眼前这个年轻人既然选择前来,那肯定是做好了觉悟,自己不便多说,但内心还是为一颗年轻的生命的即将陨落叹息。自己已经做好死的觉悟了吗?其实他也不知道。关于冰原,关于那片神秘之地的传说,关于自己的身世,他都知之甚少。他不知道自己从何而来,当他询问起此事时,别人也是一脸茫然,莫能回答,只是说,有人狩猎时在山上发现了浑浑噩噩、神志不清的他,好奇但又因害怕无法靠近,最后回到城镇找来更多的人把他带回去。值得庆幸的是,他并没有伤人的行为,只是在床上躺了两天后神志逐渐清醒,但是对于有关姓甚名谁的问题,他却不能回答。后来啊,后来他跟众人生活在一起,靠帮别人工作混口饭吃。这样的生活持续了一年,就在前几日,几个自称工作人员的人找到他,想让他去“研究”北部冰原的秘密。起初他并不想答应,毕竟他只是不清楚自己的身世,但是不是傻到白白去送死,但是那几个人承诺此行会弄清楚他的身世之谜,毕竟由于身世问题,尽管已经一起生活了许久,他与其他人之间还是有着无法消去的隔阂。他犹豫再三,最后选择了答应。\\
		\indent “我们没有别的东西能为你送行的……这块密文板……就请你收下吧”密文板是部落祖先留下的,在过去,这块密文板不知道为先辈挡了多少灾厄,长老希望,如今它也能为他带来好运。他沉默的接过密文板,简单地告别了长老,踏上了旅途……
	\section{}
		“那是什么……”在茫茫的雪原走了一整天,他看到远处似乎有缥缈的雾霭,在清晨初升的太阳的照耀下宛如淡黄色的面纱。“揭开这层面纱……就能寻找到真相”他有这个直觉,毅然决然的向远方前行。\\
		\indent 很显然,面纱后的真面目似乎并不希望自己被打搅。随着他一步一步向前迈进,风雪变得越来越大。白色的雪,因为反射阳光而熠熠生辉的冰粒,在呼啸的风的裹挟下,密密麻麻地拍在他的脸上,身上。好几次,他跌倒,重重地摔在了地上。但是想到这或许关系到更多人的未来,这或许蕴含自己身世的秘密,他又似乎找到了坚持前进的理由,爬起,匍匐到岩石后面,做完简短的休息后继续弯腰前进。\\
		\indent 时间接近正午,阳光在冰雪的反射下愈发显得刺眼。“这不是正常的现象……前方雾霭背后到底有多少秘密……”他意识到,前行……\\
		\indent “诶……风雪似乎停了……”经过数个小时的艰难跋涉,他已经不记得自己走了多远,时间过了多久,但是当身边冰雪逐渐停下的时候,他惊奇地抬起头。原来,在不知觉中,他已经来到了雾霭面前。他伸出手去探,双手刚伸进雾霭就完全看不到影踪。直觉告诉他,要穿过这片雾霭。他拖起沉重的双腿,向雾霭中跑去,但很快,他就迷失了方向,只能根据自己的直觉,继续向前跑……终于,他跑了出来,但令他惊奇的是,面前是一座大山,天空也不是他出发时的晴空,相反,天空阴沉沉的,像一盆随时都有可能倾泻下来的脏水。他抬头向山顶望去,一大团阴云笼罩在山顶上,其中时不时冒出蓝紫色的闪电,阴云似乎有生命一般,在山顶缓缓盘旋。既然已经来到这里,那自然是要解答心中的疑惑,不能半途而废。纵使心中有些许害怕,他仍然决定上山。\\
		\indent 这座山显然没有那么简单。一路上,乱木杂草横生,黑色的灌木与灰绿的杂草,一次又一次的拦住他的去路,他也一次又一次地劈开荆棘,继续前进。与此同时,尽管此处有那么多植被,却看不见任何一个鲜活的生命,并且,随着他慢慢上山,他的头开始变得越来越晕,几次差点被地上胡乱生长的树根绊倒。\\
		\indent “那是什么?!”他惊讶地发现,前方有一个洞窟,而洞窟里似乎存在本不应该存在于此的东西————几根交错而立的石柱。这一发现无疑是一针鸡血,他跑到洞窟前,用砍刀奋力地向洞窟口的冰墙砍去,霎时间,破碎的冰块轰隆轰隆地落了下来,似乎也预示着有什么尘封在冰雪之下的东西将要重见天日。他摸进洞窟,穿过稀稀疏疏交错而立的石柱。\\
		\indent 初极狭,才通人,复行数十步,豁然开朗。起初,洞窟黑暗的伸手不见五指,他只能靠伸手去摸来感受前路是否有阻碍,但当他穿过最后一簇石柱时,突如其来的亮光使得他久居黑暗的双眼不能适应。过了好一阵,他才能看清。“啊……”呈现在他眼前的,是一片古老的、由石头构成的遗迹。尽管遍地都是断垣残柱,布满青苔,结满冰块,但通过石头的大小规模来看,这曾经应该是个十分繁荣的文明。究竟是什么使得这个文明覆灭呢?他惊讶,困惑,小心翼翼的穿过断垣残柱,向里面摸索。一路上,有无数深蓝色的人形的冰,形态各异,似乎还保持着这个文明覆灭前的姿态。但其中却有不少是反着他前进的方向,似乎前方发生了令人恐惧的东西,他们想要逃离,却因为某种强大的力量而瞬间定格在了这一刻。他毛骨悚然,而随着他继续前进,他的头不仅越来越晕,还伴有耳鸣,似乎在劝退他。\\
		\indent 但这显然无法使他回去。推开门,他步入一个类似宫殿的地方,与比较敞亮的路上不同,这里颇为阴森。不知何处的角落不时传来雪落冰掉的窸窸窣窣声,每走一步,他都只能听到他脚踏在雪上的声音。突然,宫殿的大门“轰”地一声关上,四面八方传来东西在雪上移动的声音。借着缝隙透进来的光亮,他看清了,是生物,是他之前从未见过的生物,是敌人。他“唰”地一下抽出佩剑,这柄自他被发现起就在他身上的剑。
	\section{}
		经过不知多长时间的战斗,不可名状的怪物却还是源源不断地从四周涌上来。难道我今日就要命丧于此吗?就在这一分神的瞬间,不知何时来到背后的敌人一锤子将他击飞,翻滚数圈后撞上石壁停下。唉……看来没有奇迹……他望着向他靠近的敌人,闭上了眼睛,似乎已经做好准备迎接自己的结局……或许……这头痛可以让他的痛苦减轻一些吧……看来这也不完全是坏事呢……对吧……刀兵走到他面前,举起长刀……\\
		\indent 就在刀即将落下的那一瞬间,包里的密文板发出了光芒,所有的敌人动作都停了下来。嗯?怎么回事?他不可置信的睁开眼,震惊地看着从包里发出的光芒。原本来势汹汹的敌人全身慢慢结满冰晶,变成与来时路上模样相似的雕像,又在下一刻化为粉末消散在空气中。他不可置信的将密文板拿出,光芒完全褪去之后,只留下几道裂痕。\\
		\indent “呼……谢谢你啊……”暂时解决了眼前的问题,他长舒一口气,靠在墙边休息一段时间后继续前进。推开宫殿的大门,眼前是一条蜿蜒的石阶,一直向上延伸。不出意外的话,所有的真相,就埋藏在前方,但与此同时,他明白,无法预料的危险也在前方等待着他,但他别无选择,只有继续向前。顺着石阶前行,不知不觉中,他已经来到了山顶,头顶就是山顶望见的那团阴云,仍然在不断盘旋,时不时出现瘆人的闪电。而他眼前的地上,几条线交错纵横,似乎是古老的法阵。他咽下一口气,小心翼翼地靠近,在法阵中心,是一个石盘。不知从何处冒出的意识,告诉他要按下去。在他按下的瞬间,原本黑色的线发出耀眼的蓝光,头顶缓缓盘旋的阴云开始剧烈的翻滚。“你是……你为何来此……”他背后突然传来缥缈的女声,似乎来自此生无法及的远方。他瞬间回头,与此同时拔出佩剑。“你……不属于这里……你是……敌人……”眼前是一个身披白袍的女性,但是他却无法看见她的脸,又或许,她的脸本来就无法看清。她举手,霎时间,无数只能被称作“生物”的东西从法阵各处出现,向他逼近……
	\section{}
		不知战斗到了何时何刻,他的体力已经几乎到达了极限。想起了之前在破碎的宫殿发生的事情,他将包里的密文板拿了出来,高高举过头顶。霎时间,耀眼的光芒照亮山顶,除了她之外,所有的生物都停了下来,随之灰飞烟灭。“这是……”她瞬间向他飞去,等到他反应过来时,她已经来到了他的面前,随之就是一个抬手,将密文板打飞出去,摔在山石上碎成了好几块。但是或许是受到了密文板的冲击,下一刻,她捂着胸口,踉跄着后退了好几步,竭力支撑住身子。他意识到,这可能是他此生仅有的机会活下来,他用尽最后的力气,举起剑向她跑去。\\
		\indent “唔……”随着一声痛苦的呜咽,剑刺透了她的身体。“这柄剑……你……是……”我……是……我是?一绺一绺蓝光从她的伤口处钻出,钻入他的大脑,在经历宫殿一役后不在头痛的他顷刻间头痛欲裂,无数回忆涌上来……\\
		\indent ……\\
		\indent ……\\
		\indent ……\\
		\indent 遥远的北方曾经有一个繁荣的国度,这里的人们创造了灿烂的文明。这里的人们安然生活了数百年,可突然有一天,天上降下一个硕大的柱子,人们称它为“神钉”,似乎它代表了天上神明的意志。自那一天起,这个国度变得越来越寒冷,无数人无法忍受想要逃出去往南方,却发现自己已经被困在这里,一出城门就是穷凶极恶的怪物。即便已经将“神钉”拔出也无济于事,似乎这个国度已经受到了诅咒。这个国家的公主派出了无数的士兵护送骑士想要出城南下寻找解救的办法,却无一例外全部没有回来,惨死在野外。就在公主即将心灰意冷之际,一个年轻的少年主动请缨。公主别无选择,只能选择做最后的尝试,她几乎将所有的卫队都派出护送他出城。幸运的是,她这一次成功了,即便卫队只剩一个人回来报告他已出城。从那一天起,她每天早晚都登上城中最高的瞭望台,盼望着他是否回来。可是,好运似乎从来不眷顾这片土地。一场突如其来的寒流自“神钉”留下的裂隙从山顶向下席卷,城中男女老少,还没来得及反应就被定格在席卷到来的那一刻。而他们的幽魂,则在巨大的力量的作用下畸变,化作狰狞的怪物,在城中飘荡,当有外来者前来时,就显形将之残杀。公主没有立即变为怪物,却也因与寒流对抗消耗太多力气而变得奄奄一息。在最后一刻,她登上山顶的法阵,想要做最后的尝试,但是很显然,她失败了,化作了幽魂。不过与她的臣民不同的是,她没有变作怪物,因为仍留有一丝理智被封印在法阵中。\\
		\indent 她最后一刻会想些什么呢……为什么神明要如此对待这片土地上的生灵?为什么那位远去的少年迟迟未归,他是不是背叛了她?她对少年究竟持何种情感?还是……这些,如今都已无法得知……\\
		\indent 而那个少年呢?其实,他在踏出这片受诅咒的区域之时,就失去了所有有关这片领域的记忆,变作一具浑浑噩噩的躯壳,直到……
		\indent I'm always waiting for you
		\indent 她轻声说出这句话,随后消散在空中。“啊!!!”伴随一声撕心裂肺的吼声,他用尽最后的力气,把石盘劈成两半。下一刻,头顶的阴云散去,周遭的一切都化作灰消散,包括,他。
		\indent I'm coming to see you
\end{document}